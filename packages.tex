% Anpassung des Seitenlayouts ----------------------------------------------
% 	siehe Seitenstil.tex
% --------------------------------------------------------------------------
\usepackage[english]{babel}
\usepackage{bookman}
\usepackage{xfrac}
\usepackage{layout}
\usepackage{wrapfig}
\newcommand{\bsi}[2]{%
  \fontencoding{T1}\fontfamily{pbs}\fontseries{xl}\fontshape{n}%
  \fontsize{#1}{#2}\selectfont}

\renewcommand{\inghead}{\textbf{Ingredients}:\ }
\renewcommand{\rechead}{\centering\bsi{24pt}{30pt}}
\newcommand{\titel}{Moose Recipe Book}
\newcommand{\autor}{Ralf Mueller}
\newcommand{\HRule}{\rule{\linewidth}{0.5mm}}
\newcounter{wrapfigures}
\newcommand{\img}[1]
{
\begin{wrapfigure}{o}{0.6\textwidth}
 \stepcounter{wrapfigures}
  \vspace{-20pt}  
  \begin{center}
    \includegraphics[width=0.58\textwidth]{img/#1}
  \end{center}
  \vspace{-32pt}
\end{wrapfigure}
}
\makeatletter
\renewcommand*\l@subsubsection{\@dottedtocline{3}{3em}{0em}}
\makeatother
\setlength{\marginparwidth}{0pt}
\setlength\parindent{0pt}
\setlength\parskip{2ex plus 0.5ex}



% Umlaute ------------------------------------------------------------------
% 		Umlaute/Sonderzeichen wie ���� direkt im Quelltext verwenden (CodePage).
%		Erlaubt automatische Trennung von Worten mit Umlauten.
% --------------------------------------------------------------------------
\usepackage[latin1]{inputenc}
\usepackage[T1]{fontenc}
%\usepackage{ae} % "sch�neres" �
\usepackage{textcomp} % Euro-Zeichen etc.

% Grafiken -----------------------------------------------------------------
% 		Einbinden von Grafiken [draft oder final]
% 		Option [draft] bindet Bilder nicht ein - auch globale Option
% --------------------------------------------------------------------------
%\usepackage[dvips,final]{graphicx}
%\graphicspath{img/} % Dort liegen die Bilder des Dokuments

% Befehle aus AMSTeX f�r mathematische Symbole z.B. \boldsymbol \mathbb ----
\usepackage{amsmath,amsfonts}

% F�r Index-Ausgabe; \printindex -------------------------------------------
\usepackage{makeidx}

% Einfache Definition der Zeilenabst�nde und Seitenr�nder etc. -------------
\usepackage{setspace}
\usepackage{geometry}
\geometry{bindingoffset=1cm}


% Symbolverzeichnis --------------------------------------------------------
% 	Symbolverzeichnisse bequem erstellen, beruht auf MakeIndex.
% 		makeindex.exe %Name%.nlo -s nomencl.ist -o %Name%.nls
% 	erzeugt dann das Verzeichnis. Dieser Befehl kann z.B. im TeXnicCenter
%		als Postprozessor eingetragen werden, damit er nicht st�ndig manuell
%		ausgef�hrt werden muss.
%		Die Definitionen sind ausgegliedert in die Datei Abkuerzungen.tex.
% --------------------------------------------------------------------------
\usepackage[intoc]{nomencl}
  \let\abbrev\nomenclature
  \renewcommand{\nomname}{Abk�rzungsverzeichnis}
  \setlength{\nomlabelwidth}{.25\hsize}
  \renewcommand{\nomlabel}[1]{#1 \dotfill}
  \setlength{\nomitemsep}{-\parsep}


% Zum Umflie�en von Bildern -------------------------------------------------
%\usepackage{floatflt}


% Zum Einbinden von Programmcode --------------------------------------------
\usepackage{listings}
\usepackage{xcolor} 
\definecolor{hellgelb}{rgb}{1,1,0.9}
\definecolor{colKeys}{rgb}{0,0,1}
\definecolor{colIdentifier}{rgb}{0,0,0}
\definecolor{colComments}{rgb}{1,0,0}
\definecolor{colString}{rgb}{0,0.5,0}
\lstset{%
    float=hbp,%
    basicstyle=\texttt\small, %
    identifierstyle=\color{colIdentifier}, %
    keywordstyle=\color{colKeys}, %
    stringstyle=\color{colString}, %
    commentstyle=\color{colComments}, %
    columns=flexible, %
    tabsize=2, %
    frame=single, %
    extendedchars=true, %
    showspaces=false, %
    showstringspaces=false, %
    numbers=left, %
    numberstyle=\tiny, %
    breaklines=true, %
    backgroundcolor=\color{hellgelb}, %
    breakautoindent=true, %
%    captionpos=b%
}

% Lange URLs umbrechen etc. -------------------------------------------------
\usepackage{url}


% Wichtig f�r korrekte Zitierweise ------------------------------------------
\usepackage[square]{natbib}
% Quellenangaben in eckige Klammern fassen ----------------------------------
\bibpunct{[}{]}{;}{a}{}{,~}


% PDF-Optionen --------------------------------------------------------------
\usepackage[
bookmarks,
bookmarksopen=true,
pdftitle={\titel},
pdfauthor={\autor},
pdfcreator={\autor},
pdfsubject={\titel},
pdfkeywords={\titel},
colorlinks=true,
%linkcolor=red, % einfache interne Verkn�pfungen
anchorcolor=black,% Ankertext
citecolor=blue, % Verweise auf Literaturverzeichniseintr�ge im Text
filecolor=magenta, % Verkn�pfungen, die lokale Dateien �ffnen
menucolor=red, % Acrobat-Men�punkte
urlcolor=cyan, 
% f�r die Druckversion k�nnen die Farben ausgeschaltet werden:
linkcolor=black, % einfache interne Verkn�pfungen
%anchorcolor=black,% Ankertext
%citecolor=black, % Verweise auf Literaturverzeichniseintr�ge im Text
%filecolor=black, % Verkn�pfungen, die lokale Dateien �ffnen
%menucolor=black, % Acrobat-Men�punkte
%urlcolor=black, 
backref,
%pagebackref,
plainpages=false,% zur korrekten Erstellung der Bookmarks
pdfpagelabels,% zur korrekten Erstellung der Bookmarks
hypertexnames=false,% zur korrekten Erstellung der Bookmarks
%linktocpage % Seitenzahlen anstatt Text im Inhaltsverzeichnis verlinken
]{hyperref}
\usepackage{bookmark}
% Zum fortlaufenden Durchnummerieren der Fu�noten ---------------------------
\usepackage{chngcntr}


% f�r lange Tabellen
\usepackage{longtable}
\usepackage{array}
\usepackage{ragged2e}
\usepackage{lscape}

% Spaltendefinition rechtsb�ndig mit definierter Breite ---------------------
\newcolumntype{w}[1]{>{\raggedleft\hspace{0pt}}p{#1}}

% Formatierung von Listen �ndern
\usepackage{paralist}
% Standardeinstellungen:
% \setdefaultleftmargin{2.5em}{2.2em}{1.87em}{1.7em}{1em}{1em}
